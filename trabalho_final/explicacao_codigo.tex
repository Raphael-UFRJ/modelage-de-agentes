\documentclass{article}
\usepackage{amsmath}
\usepackage{listings}
\usepackage{color}

\definecolor{codegray}{gray}{0.9}
\definecolor{codegreen}{rgb}{0,0.6,0}
\definecolor{codeblue}{rgb}{0,0,0.6}

\lstdefinestyle{mystyle}{
    backgroundcolor=\color{codegray},
    commentstyle=\color{codegreen},
    keywordstyle=\color{codeblue},
    numberstyle=\tiny\color{codegray},
    stringstyle=\color{red},
    basicstyle=\ttfamily\footnotesize,
    breakatwhitespace=false,
    breaklines=true,
    captionpos=b,
    keepspaces=true,
    numbers=left,
    numbersep=5pt,
    showspaces=false,
    showstringspaces=false,
    showtabs=false,
    tabsize=2
}

\lstset{style=mystyle}

\begin{document}

\title{Explicação do Código NetLogo}
\author{}
\date{}
\maketitle

\section{Definições Iniciais}

\subsection{Declarações de Raças e Variáveis}

\begin{lstlisting}[language=NetLogo]
breed [students student]

turtles-own [
  stage
  time-in-stage
  next-stage
  total-time
]

globals [
  stages
  stage-names
  total-times
  total-counts
  average-times
  deviation
  total-students
  completion-times
]
\end{lstlisting}

\begin{itemize}
    \item \texttt{breed [students student]}: Define uma nova raça chamada "students" (alunos) com indivíduos do tipo "student".
    \item \texttt{turtles-own}: Define atributos específicos dos alunos, como estágio atual, tempo no estágio, próximo estágio e tempo total.
    \item \texttt{globals}: Define variáveis globais, como lista de estágios (\texttt{stages}), nomes dos estágios (\texttt{stage-names}), tempos totais em cada estágio (\texttt{total-times}), contagem de alunos em cada estágio (\texttt{total-counts}), tempos médios (\texttt{average-times}), desvio padrão (\texttt{deviation}), total de alunos (\texttt{total-students}) e tempos de conclusão (\texttt{completion-times}).
\end{itemize}

\section{Procedimento \texttt{setup}}

\begin{lstlisting}[language=NetLogo]
to setup
  clear-all
  reset-ticks
  set stages (list
    patch -10 -16  ;; ENTRY
    patch -15 -8   ;; PAY
    patch -16 2    ;; B1
    patch -16 10   ;; B2
    patch -10 16   ;; B3
    patch -2 16    ;; B4
    patch 8 16     ;; B5
    patch 15 10    ;; B6
    patch 15 -2    ;; B7
    patch 15 -8    ;; EXIT
  )
  set stage-names ["ENTRY" "PAY" "B1" "B2" "B3" "B4" "B5" "B6" "B7" "EXIT"]
  set total-times n-values (length stages) [0]
  set total-counts n-values (length stages) [0]
  set average-times n-values (length stages) [0]
  set total-students 0
  ask patches [set pcolor 66]
  set completion-times []
  create-sector
  create-public initial-group-size
end
\end{lstlisting}

\begin{itemize}
    \item \texttt{setup}: Inicializa o ambiente de simulação.
    \begin{itemize}
        \item \texttt{clear-all} e \texttt{reset-ticks}: Limpa o ambiente e reinicia o contador de ticks.
        \item \texttt{set stages}: Define a lista de patches que representam os estágios do processo.
        \item \texttt{set stage-names}: Define os nomes dos estágios.
        \item \texttt{set total-times}, \texttt{total-counts}, \texttt{average-times}: Inicializa listas para armazenar os tempos totais, contagens e tempos médios.
        \item \texttt{set total-students}: Inicializa a contagem total de alunos.
        \item \texttt{ask patches [set pcolor 66]}: Define a cor dos patches.
        \item \texttt{set completion-times}: Inicializa a lista de tempos de conclusão.
        \item \texttt{create-sector}: Cria a representação dos setores.
        \item \texttt{create-public initial-group-size}: Cria um grupo inicial de alunos.
    \end{itemize}
\end{itemize}

\section{Procedimento \texttt{create-sector}}

\begin{lstlisting}[language=NetLogo]
to create-sector
  foreach stages [
    stage-patch ->
    let stage-index position stage-patch stages
    ask stage-patch [
      set plabel item stage-index stage-names
    ]
  ]
end
\end{lstlisting}

\begin{itemize}
    \item \texttt{create-sector}: Associa os patches dos estágios aos nomes correspondentes.
    \begin{itemize}
        \item \texttt{foreach stages}: Itera sobre cada patch na lista de estágios.
        \item \texttt{ask stage-patch [set plabel item stage-index stage-names]}: Define o rótulo do patch com o nome do estágio correspondente.
    \end{itemize}
\end{itemize}

\section{Procedimento \texttt{create-public}}

\begin{lstlisting}[language=NetLogo]
to create-public [initial-size]
  create-students initial-size [
    set shape "person"
    setxy -11 -16
    set stage item 0 stages
    set time-in-stage 0
    set next-stage nobody
    set total-time 0
  ]
end
\end{lstlisting}

\begin{itemize}
    \item \texttt{create-public}: Cria alunos.
    \begin{itemize}
        \item \texttt{create-students initial-size}: Cria um número inicial de alunos.
        \item \texttt{set shape "person"}: Define a forma dos alunos como "person".
        \item \texttt{setxy -11 -16}: Posiciona os alunos no ponto inicial.
        \item \texttt{set stage item 0 stages}: Define o estágio inicial dos alunos.
        \item \texttt{set time-in-stage 0}, \texttt{set next-stage nobody}, \texttt{set total-time 0}: Inicializa os atributos dos alunos.
    \end{itemize}
\end{itemize}

\section{Procedimento \texttt{go}}

\begin{lstlisting}[language=NetLogo]
to go
  if not any? students [
    display-average-times
    stop
  ]
  move-students
  tick
  display-average-times
  possibly-create-new-students
end
\end{lstlisting}

\begin{itemize}
    \item \texttt{go}: Principal ciclo de execução.
    \begin{itemize}
        \item \texttt{if not any? students}: Verifica se não há mais alunos.
        \item \texttt{display-average-times}: Exibe os tempos médios.
        \item \texttt{stop}: Para a simulação.
        \item \texttt{move-students}: Move os alunos.
        \item \texttt{tick}: Incrementa o contador de ticks.
        \item \texttt{possibly-create-new-students}: Possivelmente cria novos alunos.
    \end{itemize}
\end{itemize}

\section{Procedimento \texttt{move-students}}

\begin{lstlisting}[language=NetLogo]
to move-students
  ask students [
    let current-index position stage stages
    ifelse current-index = (length stages - 1) [
      set total-time total-time + time-in-stage ;; Acumula o tempo gasto na última etapa
      set completion-times lput total-time completion-times ;; Armazena o tempo de conclusão
      set total-students total-students + 1
      show (word "Tempo completado: " completion-times)
      die
    ] [
      set next-stage item (current-index + 1) stages
      if current-index < (length stages - 1) [
        update-stage-time current-index time-in-stage
      ]
      ifelse count students-on next-stage = 0 and time-in-stage >= random-triangular 0.1 move-threshold 10.8 [
        move-to next-stage
        set stage next-stage
        set time-in-stage 0
        set total-time total-time + time-in-stage
      ] [
        set time-in-stage time-in-stage + 1
      ]
    ]
  ]
end
\end{lstlisting}

\begin{itemize}
    \item \texttt{move-students}: Move os alunos pelos estágios.
    \begin{itemize}
        \item \texttt{ifelse current-index = (length stages - 1)}: Verifica se o aluno está no último estágio.
        \begin{itemize}
            \item \texttt{set total-time total-time + time-in-stage}: Acumula o tempo gasto na última etapa.
            \item \texttt{set completion-times lput total-time completion-times}: Armazena o tempo de conclusão.
            \item \texttt{set total-students total-students + 1}: Incrementa o total de alunos.
            \item \texttt{show (word "Tempo completado: " completion-times)}: Exibe os tempos de conclusão.
            \item \texttt{die}: Remove o aluno.
        \end{itemize}
        \item \texttt{set next-stage item (current-index + 1) stages}: Define o próximo estágio.
        \item \texttt{if current-index < (length stages - 1)}: Verifica se o aluno não está no último estágio.
        \item \texttt{update-stage-time current-index time-in-stage}: Atualiza o tempo gasto no estágio.
        \item \texttt{ifelse count students-on next-stage = 0 and time-in-stage >= random-triangular 0.1 move-threshold 10.8}: Verifica se pode mover para o próximo estágio.
        \begin{itemize}
            \item \texttt{move-to next-stage}: Move para o próximo estágio.
            \item \texttt{set stage next-stage}: Define o próximo estágio.
            \item \texttt{set time-in-stage 0}: Reinicia o tempo no estágio.
            \item \texttt{set total-time total-time + time-in-stage}: Acumula o tempo total.
        \end{itemize}
        \item \texttt{set time-in-stage time-in-stage + 1}: Incrementa o tempo no estágio.
    \end{itemize}
\end{itemize}

\section{Procedimento \texttt{possibly-create-new-students}}

\begin{lstlisting}[language=NetLogo]
to possibly-create-new-students
  if random-float 1 < new-student-prob [
    create-public 1
  ]
end
\end{lstlisting}

\begin{itemize}
    \item \texttt{possibly-create-new-students}: Possivelmente cria novos alunos.
    \begin{itemize}
        \item \texttt{if random-float 1 < new-student-prob}: Verifica se deve criar um novo aluno.
        \item \texttt{create-public 1}: Cria um novo aluno.
    \end{itemize}
\end{itemize}

\section{Procedimento \texttt{update-stage-time}}

\begin{lstlisting}[language=NetLogo]
to update-stage-time [index time]
  set total-times replace-item index total-times (item index total-times + time)
  set total-counts replace-item index total-counts (item index total-counts + 1)
end
\end{lstlisting}

\begin{itemize}
    \item \texttt{update-stage-time}: Atualiza o tempo gasto em um estágio.
    \begin{itemize}
        \item \texttt{set total-times replace-item index total-times (item index total-times + time)}: Atualiza o tempo total no estágio.
        \item \texttt{set total-counts replace-item index total-counts (item index total-counts + 1)}: Atualiza a contagem de alunos no estágio.
    \end{itemize}
\end{itemize}

\section{Procedimento \texttt{average-time}}

\begin{lstlisting}[language=NetLogo]
to-report average-time [total-time-val total-count-val]
  ifelse total-count-val > 0 [
    report total-time-val / total-count-val
  ] [
    report 0
  ]
end
\end{lstlisting}

\begin{itemize}
    \item \texttt{average-time}: Calcula o tempo médio em um estágio.
    \begin{itemize}
        \item \texttt{ifelse total-count-val > 0}: Verifica se há alunos no estágio.
        \item \texttt{report total-time-val / total-count-val}: Retorna o tempo médio.
        \item \texttt{report 0}: Retorna 0 se não há alunos.
    \end{itemize}
\end{itemize}

\section{Procedimento \texttt{display-average-times}}

\begin{lstlisting}[language=NetLogo]
to display-average-times
  let temp-average-times []
  (foreach stage-names total-times total-counts [
    [stage-name total-time-val total-count-val] ->
      let avg-time average-time total-time-val total-count-val
      set temp-average-times lput avg-time temp-average-times
      show (word "Tempo medio em " stage-name ": " avg-time " (total time: " total-time-val ", total count: " total-count-val ")")
  ])
  set average-times temp-average-times
  show (word "Lista de tempos medios: " average-times)
end
\end{lstlisting}

\begin{itemize}
    \item \texttt{display-average-times}: Exibe os tempos médios em cada estágio.
    \begin{itemize}
        \item \texttt{let temp-average-times []}: Inicializa uma lista temporária para armazenar os tempos médios.
        \item \texttt{foreach stage-names total-times total-counts}: Itera sobre os nomes dos estágios, tempos totais e contagens.
        \item \texttt{let avg-time average-time total-time-val total-count-val}: Calcula o tempo médio.
        \item \texttt{set temp-average-times lput avg-time temp-average-times}: Adiciona o tempo médio à lista temporária.
        \item \texttt{show (word "Tempo medio em " stage-name ": " avg-time " (total time: " total-time-val ", total count: " total-count-val ")")}: Exibe o tempo médio para cada estágio.
        \item \texttt{set average-times temp-average-times}: Atualiza a lista global de tempos médios.
        \item \texttt{show (word "Lista de tempos medios: " average-times)}: Exibe a lista de tempos médios.
    \end{itemize}
\end{itemize}

\section{Procedimento \texttt{random-triangular}}

\begin{lstlisting}[language=NetLogo]
to-report random-triangular [min-val max-val mode]
  let u1 random-float 1
  let u2 random-float 1
  let f (mode - min-val) / (max-val - min-val)
  ifelse u1 <= f [
    report min-val + sqrt (u2 * (max-val - min-val) * (mode - min-val))
  ] [
    report max-val - sqrt ((1 - u2) * (max-val - min-val) * (max-val - mode))
  ]
end
\end{lstlisting}

\begin{itemize}
    \item \texttt{random-triangular}: Gera um número aleatório com distribuição triangular.
    \begin{itemize}
        \item \texttt{let u1 random-float 1}, \texttt{let u2 random-float 1}: Gera dois números aleatórios entre 0 e 1.
        \item \texttt{let f (mode - min-val) / (max-val - min-val)}: Calcula a fração da base da distribuição triangular.
        \item \texttt{ifelse u1 <= f}: Verifica se o número aleatório está na parte ascendente da distribuição.
        \item \texttt{report min-val + sqrt (u2 * (max-val - min-val) * (mode - min-val))}: Retorna um número na parte ascendente.
        \item \texttt{report max-val - sqrt ((1 - u2) * (max-val - min-val) * (max-val - mode))}: Retorna um número na parte descendente.
    \end{itemize}
\end{itemize}

\end{document}
